100 hojas máximo
Capitulo Tecnologías (15 hojas)
Lenguajes de programación.
Diferencia entre lenguajes Web. Lenguajes escritorio, Celulares …..
Elegiste XXXXXXXX

El lenguaje de programación web PHP [1] ccccccccccccccccccccccccccccccccccccccccccccccccccccccccccccccccccccccccccc

Base de datos
	Mysql
	Postgrade
	Oracle
Patrones de Análisis y Diseño Teoría
Modelos de Dominio, Casos de uso

Capitulo  Administración Financiero
Herramientas 6 meses
Capitulo COSTOS de hacer estas herramienta  (15 hojas)
Programadores, …….
Análisis costos y recuperación


Capitulo Describir la Herramientas (15 hojas)
Modelo de clase para el dominio del problema
Modelo de CU (funcionalidades)
	Evaluar
Editar
Patrones que utilizaste

Capitulo Caso de Estudio (10 hojas)

TRABAJANDO:
BLOG “TTTTTTTTTTTTTTT”   nooooo
WiKIPEDIA   nooooooo

REFERENCIAS BIBLIOGRAFICAS (5 HOJAS)
[1] PÁGINA OFICIAL DE PHP    
[2] Libro de Ingeniería del software de …..
cAPITULO MÉTODO DE EVALUACIÓN
LSP
(10 HOJAS)
CARATULA 1 HOJAS
INDICE 3 HOJAS

Introduction:

    Briefly introduce the purpose of the technology chapter and its significance in the context of your paper.

Comparing Programming Languages:
For each programming language (Java, C, Python, PHP, JavaScript, SQL), consider including the following aspects in your comparison:

    Syntax and Structure:
        Discuss the syntax and general structure of the language.
        Highlight any unique features that stand out.

    Use Cases and Domains:
        Describe the typical use cases and domains where the language is commonly used.
        Explain how each language's strengths align with different types of software development projects.

    Ease of Learning and Readability:
        Evaluate how easy the language is to learn for engineers with different backgrounds.
        Comment on the readability of the code and its impact on collaboration.

    Performance and Efficiency:
        Compare the runtime performance and efficiency of each language.
        Discuss any optimizations or drawbacks associated with each language.

    Ecosystem and Libraries:
        Explore the availability of libraries, frameworks, and tools that support each language.
        Discuss how these resources contribute to development productivity.

    Community and Support:
        Consider the size and activity of the language's developer community.
        Discuss the availability of documentation, forums, and online resources.

Selection of Java:
Explain why Java was chosen as the selected programming language for the project. Consider the following points:

    Compatibility: Discuss how Java's "write once, run anywhere" philosophy aligns with the project's goals.
    Mature Ecosystem: Highlight Java's extensive libraries and frameworks that can aid engineers.
    Performance: Explain how Java's compiled nature contributes to performance and stability.
    Industry Adoption: Discuss Java's widespread use in enterprise-level software development.
    Team Familiarity: If applicable, mention any prior experience your team has with Java.

Conclusion:
Sum up the key takeaways from your comparison and selection of Java. Emphasize how the chosen programming language aligns with the goals of the software tool and the needs of engineers
<


Biographical Information:

    Java:
        Java was developed by James Gosling at Sun Microsystems and was first released in 1995.
        It is known for its platform independence, object-oriented nature, and strong emphasis on readability.
        The language's tagline "write once, run anywhere" speaks to its portability.

    C:
        C was created by Dennis Ritchie at Bell Labs in the early 1970s.
        It is considered one of the foundational programming languages and has influenced many others.
        C's direct memory manipulation capabilities make it popular for systems programming.

    Python:
        Python was created by Guido van Rossum and was first released in 1991.
        Known for its simplicity and readability, Python emphasizes code readability and a clear syntax.
        It has gained popularity for web development, data analysis, and automation.

    PHP:
        PHP, originally created by Rasmus Lerdorf in 1994, stands for "PHP: Hypertext Preprocessor."
        It was designed for web development and server-side scripting, making it a cornerstone of web applications.

    JavaScript:
        JavaScript, developed by Brendan Eich at Netscape, was introduced in the mid-1990s.
        Despite its name, JavaScript has evolved into a versatile scripting language used for both frontend and backend web development.

    SQL:
        SQL (Structured Query Language) has roots dating back to the 1970s.
        It is specialized for managing and querying relational databases, playing a crucial role in data storage and retrieval.

Additional Information:

    Popularity and Usage:
        Consider mentioning the TIOBE index or other metrics that indicate the popularity of these languages.
        Discuss which languages are favored in different application domains (e.g., Java in Android development, Python in data science).

    Language Evolution:
        Highlight how each language has evolved over time, adapting to changing trends and technological advancements.

    Community and Collaboration:
        Emphasize the importance of strong developer communities and how they contribute to language development and support.

    Case Studies:
        If possible, include brief case studies or examples of notable projects that used each programming language.

    Crossover and Integration:
        Mention instances where these languages are used together or integrated, such as Python's integration with C/C++ for performance optimization.

    Future Prospects:
        Provide insights into the future of these languages, considering factors like emerging technologies and industry trends.
